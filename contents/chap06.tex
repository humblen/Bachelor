% 本文件是示例论文的一部分
% 论文的主文件位于上级目录的 `main.tex`

\chapter{总结与展望}

\section{本文工作总结}
本文基于YOLOv11算法,针对摩托车驾乘人员头盔佩戴数据集训练了两个模型:头盔佩戴情况检测模型和驾驶员检测模型,设计并实现了一套高效的检测系统,能够帮助执法人员减少人工工作量,推动交通执法向智能化方向发展。

在数据集构建过程中,对原始数据进行了细致处理,通过对原样本数据做图像增强保证每个类别和驾驶员都有足够数量的样本图片,采用7:2:1的比例对数据集进行随机划分,分别构建训练集、验证集和测试集。通过YOLO训练过程中的不同指标数据来对比不同模型的检测性能。

本文基于YOLOv11n、YOLOv11s和YOLOv11m三个不同权重的模型进行了训练并分析了结果。实验表明,三个模型在精度上都表现出色,均在96\%以上,YOLOv11m的精度最高,达到了97.3\%。召回率也都在96\%以上,且也是YOLOv11m的召回率最高,为98.1\%。三个模型的mAP50值几乎一致,但mAP50-95指标YOLOv11和YOLOv11m最好,说明这两个模型在更严格的IoU阈值下性能更好。三个模型的检测速度从YOLOv11n、YOLOv11s到YOLOv11m依次变慢。YOLO模型的检测精度和检测速度整体上是呈现负相关的,提升模型的检测精度会导致模型检测速度有下降。结合检测精度和检测速度两个方面各个指标来看,如果追求高检测速度,YOLOv11n模型是比较好的选择;如果要求严格的检测精度,YOLOv11m更合适,但是会牺牲一些检测速度;YOLOv11s是介于以上两个模型之间比较适中的选择。

本文基于浏览器/服务器架构进行前后端分离的系统开发。前端为用户提供了目标检测和记录查询功能;后端则高效处理前端请求,完成图片或视频的检测、驾驶员信息识别以及结果存储等功能。同时,系统支持用户自定义检测模型、IoU 和置信度参数,历史检测结果可通过多种可视化图表呈现,满足了交管部门对数据的分析需求。

\section{研究展望}
随着科技的不断发展,目标检测技术在智能交通领域将迎来更广阔的发展空间。尽管YOLOv11模型在本文中表现优秀,但也存在着提升空间。

在模型性能方面,可以引入注意力机制、自监督学习等技术,增强模型对复杂场景和小目标的检测能力;针对头盔颜色与背景或衣服相近导致识别困难的问题,可扩充训练数据集,收集大量此类特殊样本,使模型学习对应特征模式,同时对现有数据进行颜色扰动增强,模拟多种相近情况,提升数据多样性。

在数据集方面,未来可收集更多不同场景下的摩托车驾乘人员数据,包括不同天气条件、不同光照环境、不同地域的交通数据等,增强模型的泛化能力。根据本文的实验结果,可以看出YOLO模型在检测精度和速度上面呈现负相关性,在提升精度的同时,不可避免地会损失一些速度,反之亦然。随着深度学习技术的不断突破、硬件计算能力的不断提升,YOLO有望实现速度和精度的协同突破,让YOLO模型既能在复杂场景中精准识别目标,又能以极快的速度完成实时检测任务。

在应用与落地方面,本文实现的摩托车驾乘人员头盔佩戴检测系统未来可广泛应用于智能交通监管,接入道路上的监控摄像头,实时检测驾乘人员的头盔佩戴情况,将违规驾驶员通知给执法部门,联动执法。

本文主要研究了摩托车驾乘人员头盔佩戴情况的检测系统,未来可将其拓展到其他领域,如汽车安全带佩戴检测、机动车违规行为检测等。同时,还可以与其他模块(如交通流量监测、违章行为自动抓拍等)进行深度融合,助力智能交通系统的发展。

%%% Local Variables: 
%%% mode: latex
%%% TeX-master: "../main.tex"
%%% End:
