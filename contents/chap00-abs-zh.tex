随着经济的发展与出行需求的增长,摩托车凭借其便捷性,保有量不断增加。但由于交通管理难度增大、部分驾驶员安全意识不足等因素,涉及摩托车的交通事故也逐渐增多。头盔佩戴与否直接关系到驾乘人员在事故中的伤亡程度。本文旨在设计并实现一套摩托车驾乘人员头盔佩戴检测系统,实时检测驾乘人员的头盔佩戴情况,并将检测结果以数据可视化的形式呈现。本文的主要工作涵盖检测模型训练和系统开发两个重要部分。

检测模型训练方面,基于YOLO目标识别算法,构建了双模型协同架构:其中,第一个模型聚焦于快速、准确地识别图像或视频中摩托车及其驾乘人员整体的头盔佩戴状态;第二个模型则以前者输出的关键区域作为输入,精准定位并识别驾驶人员的id。在研究过程中,对YOLOv11n、YOLOv11s和YOLOv11m三个不同规模的模型的训练效果进行了全面对比分析,并总结了各个模型所适用的检测场景。

系统开发方面,基于浏览器/服务器(B/S)架构开发前后端分离的系统,使用Vue框架搭建B端页面,SpringBoot框架搭建S端服务器。前端浏览器为用户提供了头盔佩戴检测和历史记录查询两大核心功能。头盔佩戴检测功能允许用户根据自身需求自定义检测模型、检测交并比(IoU)和置信度,同时支持对检测结果的浏览和保存操作。历史记录查询功能则支持用户根据特定条件查询检测记录,并通过可视化图表的形式辅助用户进行数据分析。后端服务器主要分为目标检测和记录查询两个模块,分别与前端的两大功能进行对接。目标检测模块调用模型对用户上传的图片或视频进行检测,并存储检测结果;记录查询模块根据用户输入的过滤条件,对数据库中的历史记录进行查询并返回结果。