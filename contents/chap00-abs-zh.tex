随着经济的发展与出行需求的增长,摩托车凭借其便捷性,保有量不断增加。但由于交通管理难度增大、部分驾驶员安全意识不足等因素,涉及摩托车的交通事故也逐渐增多。头盔佩戴与否直接关系到驾乘人员在事故中的伤亡程度。本文旨在设计并实现一套摩托车驾乘人员头盔佩戴检测系统,实时检测驾乘人员的头盔佩戴情况,并对检测结果实现数据可视化。本文主要工作包含检测模型训练和系统开发两个部分。

检测模型训练方面,基于YOLO目标识别算法,构建了双模型协同架构:其一,专注于快速识别图像或视频中摩托车及其驾乘人员整体的头盔佩戴状态;其二,以前者输出的关键区域为输入,精准定位并识别驾驶人员id。研究过程中,系统对比分析了YOLOv11n、YOLOv11s和YOLOv11m三个不同规模模型的训练效果,总结了不同模型适用的检测场景。

系统开发方面,基于浏览器/服务器(B/S)架构开发前后端分离的系统,使用Vue框架搭建B端页面,SpringBoot框架搭建S端服务器。前端浏览器为用户提供头盔佩戴检测和历史记录查询两大功能。头盔佩戴检测功能允许用户自定义检测模型、检测交并比(IoU)和置信度,支持对检测结果的浏览和保存。历史记录查询功能支持用户对查询字段的过滤以及数据可视化。后端服务器分为目标检测和记录查询两个模块,分别用来对接前端的两大功能。目标检测模块接收用户输入的数据并调用模型进行检测,最后返回检测结果。记录查询模块根据用户输入的过滤条件查询数据库中的历史记录并返回。